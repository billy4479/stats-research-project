\documentclass[11pt]{extarticle}

\title{Does commuting affect quality of life?}
\author{Ellero Giacomo, Porta Beatrice}
\date{a.y. 2024/2025}

\usepackage{parskip} % for no indentation
\usepackage[a4paper,margin=1.5cm]{geometry} % to adjust margins
\usepackage{bookmark} % for pdf bookmarks

\usepackage{enumitem} % for custom lists
\usepackage{float}
\usepackage{dirtytalk}
\usepackage{array}
\usepackage{makecell}
\usepackage{subcaption}

% We want numbers on subsections too
\setcounter{secnumdepth}{3}

\usepackage{amssymb}   % for varnothing and other weirds symbols
\usepackage{amsmath}   % basically everything
\usepackage{amsthm}    % for proof environment
\usepackage{mathtools} % for underbrace, arrows, and a lot of other things
\usepackage{mathrsfs}  % for "mathscr"
\usepackage{bm}        % for bold math symbols
\usepackage{physics}   % for derivatives and lots of operators
\usepackage{dsfont}    % for \mathds{1}

\usepackage{xcolor}
\usepackage{cleveref}
\usepackage{hyperref}
\hypersetup{
    colorlinks=true,
    linkcolor=teal,
    urlcolor=blue,
}
\urlstyle{same}

\numberwithin{table}{section}
\numberwithin{figure}{section}

\usepackage{cancel}    % for canceling terms
\renewcommand{\CancelColor}{\color{red}}

% Operators
\newcommand{\gt}{>}
\newcommand{\lt}{<}
\newcommand{\indep}{\perp \!\!\! \perp}

% Sets
\newcommand{\C}{\mathds{C}}
\newcommand{\R}{\mathds{R}}
\newcommand{\N}{\mathds{N}}
\newcommand{\Q}{\mathds{Q}}
\newcommand{\Z}{\mathds{Z}}

\numberwithin{equation}{section}

\begin{document}

\maketitle
%\tableofcontents

\section{Introduction}

With this research we want to investigate whether there’s a difference in the academic performances
and the quality of life between students who are commuters and those who aren't.
Commuters are those people who work or study relatively far from where they live
and need to travel back and forth from their workplace/university every day.

The motive behind our research sparked because both the researchers are commuters,
alongside the majority of their friends, while in their class at Bocconi university most students
live very close to the university.
We wondered if living in the city where their University is located could be beneficial
to their well-being and also academic performance.

The intuition we had is that the time spent commuting could be better invested in studying
or doing other activities more beneficial to the overall well-being.
Moreover, public transports are often a source of stress, due to delays, strikes and discomfort.

However, the benefits of not having to commute are all mere speculations without any data to prove
that there’s an actual correlation.

All the data we collected and the scripts we used are available at
\href{https://github.com/billy4479/stats-research-project}{the following GitHub repository}.

\section{Data collection}

As we didn't find any existing dataset on this specific topic of research we decided
to collect our own data through a survey.

The survey, both in Italian and in English, was sent to many students all over Italy.
We managed to collect about $100$ answers, with about $60\%$ of replies being from commuters.

This is probably the greatest limitation of our project:
the sample is very small and not very diverse, which does not help providing accurate results.
Moreover, many of the respondents were unable to provide a GPA,
so that reduced the available data even further.

After collecting the data we passed it through the \texttt{merge\_data.py} script,
which merged the results from the Italian and English form and expressed the results in a more
\say{statistic friendly} way.

The complete data can be found at
\url{https://github.com/billy4479/stats-research-project/blob/master/data/merged.csv}.
The final dataframe was organized accordingly to \cref{table:data}.

\begin{table}[!ht]
	\begin{center}
		\begin{tabular}{|c|c|c|c|}
			\hline
			\textbf{n} & \textbf{Variable}                & \textbf{Type}                  & \textbf{Description}                           \\
			\hline
			\hline
			0          & \texttt{university}              & string                         & Name of the university attended.               \\
			\hline
			1          & \texttt{age}                     & integer                        & Age.                                           \\
			\hline
			2          & \texttt{is\_commuter}            & boolean                        & Considers themselves a commuter.               \\
			\hline
			3          & \texttt{commute\_time}           & integer $\in \{1, \dots, 5\}$  & Time taken to go to university.                \\
			\hline
			4          & \texttt{attendance}              & integer $\in \{1, \dots, 4\}$  & Percentage of classes attended.                \\
			\hline
			5          & \texttt{gpa}                     & real $\in [0,30]$              & GPA.                                           \\
			\hline
			6          & \texttt{did\_move}               & boolean                        & Did move to attend university.                 \\
			\hline
			7          & \texttt{means\_of\_transport}    & list of strings                & List of means of transport used.               \\
			\hline
			8          & \texttt{cor\_commute\_study}     & integer $\in \{1, \dots, 10\}$ & \makecell{Expects that commuting influences    \\negatively studies.}              \\
			\hline
			9          & \texttt{no\_study\_time}         & integer $\in \{1, \dots, 10\}$ & Doesn't have enough time to study.             \\
			\hline
			10         & \texttt{higher\_gpa\_if\_closer} & integer $\in \{1, \dots, 10\}$ & \makecell{Thinks they would have a better GPA  \\ if they lived closer to university.} \\
			\hline
			11         & \texttt{no\_hobbies}             & integer $\in \{1, \dots, 10\}$ & \makecell{Doesn't have enough time to spend on \\ hobbies and sports.} \\
			\hline
			12         & \texttt{stress}                  & integer $\in \{1, \dots, 10\}$ & Going to university is stressful.              \\
			\hline
			13         & \texttt{no\_sleep}               & integer $\in \{1, \dots, 10\}$ & Does not get enough sleep.                     \\
			\hline
			14         & \texttt{no\_friends}             & integer $\in \{1, \dots, 10\}$ & Not enough time to see their friends.          \\
			\hline
			15         & \texttt{no\_family}              & integer $\in \{1, \dots, 10\}$ & \makecell{Not enough time to spend             \\ with their family.}    \\
			\hline
			16         & \texttt{loneliness}              & integer $\in \{1, \dots, 10\}$ & Feels lonely.                                  \\
			\hline
		\end{tabular}
		\caption{
			Data collected through the survey.
			Integer values have been normalized in order to fint in the interval $[0,1] \in \R$.
			The data with indices $9$ through $16$ inclusive is what we called Quality of Life Indicators (QLI).
		}
	\end{center}
	\label{table:data}
\end{table}

\section{Tests}

\section{Regressions}

\subsection{Assumptions}

Linear regressions make some assumptions about the data which have to be checked before
running the model.

\subsubsection{Normality}

First we have to verify that the independent variable follows a normal distribution.
We performed Shapiro-Wilk test on the GPA and all the QLIs which told us
with very high confidence that our data is not normally distributed.
We also made some QQ-plots to confirm graphically the results of the test.

\begin{figure}[!ht]
	\centering
	\subcaptionbox{QQ-plot for the \texttt{gpa}}
	{\includegraphics[width=0.3\textwidth]{./routput/gpa_regression/qqplot_gpa.pdf}}
	\hspace{1cm}
	\subcaptionbox{QQ-plot for \texttt{stress}}
	{\includegraphics[width=0.3\textwidth]{./routput/qli_regression/qqplot_stress.pdf}}
	\caption{QQ-plot of \texttt{gpa} and \texttt{stress}
		(taken as an example within the QLIs, other QLIs show a similar behavior).
	}
	\label{fig:qq-gpa-stress}
\end{figure}

As can be seen from the plots in \cref{fig:qq-gpa-stress},
our data is neither very normal nor very continuous:
it is clearly possible to see that there were $10$ options available in the QLI survey
and that most people wrote an integer GPA.

This is probably due to the very small sample size
and the samples are not \textit{too} far away from the confidence region of the QQ-plot.
However this is definitely something to keep in mind when
evaluating the effectiveness of our regressions.

\subsubsection{Correlation}
\label{sec:correlation}

The next step we took was to verify that our predictors
were not too correlated with each other as this can negatively impact the performance
of the regression.

\begin{figure}[!ht]
	\centering
	\includegraphics[width=0.5\textwidth]{./routput/correlation/correlation.pdf}
	\caption{Correlation matrix for all the predictors used in the regressions.}
	\label{fig:corr}
\end{figure}

As it is possible to see in \cref{fig:corr}, there are some predictors which
are very correlated with each other.
From this matrix we can extract the following information:
\begin{itemize}
	\item \texttt{commute\_time}, \texttt{is\_commuter} and \texttt{did\_move}
	      are all very correlated;
	\item \texttt{is\_commuter} is positively correlated to the use of public transport
	      and in particular to the use of trains, while negatively to going by foot;
	\item The use of the metro is positively correlated to
	      the number of means used and the use of public transport at all.
\end{itemize}

From these results we decided to exclude \texttt{is\_commuter} and \texttt{did\_move}
from the regressions as they are too correlated to \texttt{commute\_time},
which also carries more information than the other two.

\subsubsection{Residuals}

To complete the checks on the assumptions of the linear regressions
we also checked that the residuals are normally distributed with constant variance.

Both these assumptions where verified for all regressions,
the first one by using Shapiro-Wilk tests and QQ-plots, the second one using scatter plots (see \cref{fig:residuals}).

\begin{figure}[!ht]
	\centering
	\subcaptionbox{QQ-plot for the residuals in the GPA regression.}
	{\includegraphics[width=0.3\textwidth]{./routput/gpa_regression/qqplot_residuals.pdf}}
	\hspace{1cm}
	\subcaptionbox{Scatter plot for residuals in \texttt{no\_sleep} regression.}
	{\includegraphics[width=0.3\textwidth]{./routput/qli_regression/scatter_residuals_no_sleep.pdf}}
	\caption{A QQ-plot and a scatter plot of the residuals in two different regressions.
		These graphs were taken as examples:
		in other regressions we obtained the same results.}
	\label{fig:residuals}
\end{figure}

\subsection{Linear regression on GPA}

First we tried to do a linear regression to try to estimate the GPA
based on the predictors we have available.

We started by using almost all the predictors we had available,
from there we started removing those predictors that were correlated with each other,
according to \cref{sec:correlation}, and to reduce the possibility of overfitting
we also performed a stepwise regression to eliminate predictors in excess.

We ran the stepwise regression both forwards and backwards,
using the AICc penalty criterion, which is a modified version of AIC with an added penalty term
for the size of the dataset.
This was particularly useful since small datasets are more prone to overfitting.
\begin{equation}
	{\operatorfont AICc} =
	\underbrace{2k + 2 \ln (\hat L)}_{\operatorfont AIC}
	+ \underbrace{\frac {2k^{2}+2k}{n-k-1}}_{\text{extra penalty term}}
\end{equation}

We algorithm settled at a model containing the following predictors:
\texttt{commute\_time},
\texttt{higher\_gpa\_if\_closer},
\texttt{no\_study\_time},
\texttt{no\_hobbies},
\texttt{stress},
\texttt{no\_sleep},
\texttt{no\_family},
\texttt{no\_friends},
\texttt{loneliness}, and
\texttt{attendance}.

Running the linear regression with these predictors gave as result that almost none of them
are significantly different from $0$, with the exception of
\texttt{commute\_time} and \texttt{loneliness} which both had a $p$-value of $0.07$.
The model had a very low $R^2$ of around $0.25$.

However, running an analysis of variance on the model showed that both
\texttt{commute\_time} and \texttt{no\_study\_time} are significant
at explaining the variance of the model, with both $p$-values $<0.05$.

\subsection{Linear regression on each QLI}

Next we tried to predict each Quality of Life Indicator.
We used the same stepwise algorithm on each indicator we wanted to predict,
which gave us a moderately size model for the linear regression.

The full model we started from used the following predictors:
\texttt{commute\_time},
\texttt{is\_commuter}, and the following variables derived from \texttt{means\_of\_transport}:
\texttt{did\_move},
\texttt{use\_foot},
\texttt{use\_bike},
\texttt{use\_bus},
\texttt{use\_metro},
\texttt{use\_tram},
\texttt{use\_train},
\texttt{use\_car}, and
\texttt{n\_means\_used}.

Here again all regressions have a very low $R^2$ value, hovering around $0.2$ for all QLIs.
We will report just the most notable findings between all these regressions:
\begin{itemize}
	\item TODO
\end{itemize}

\section{Conclusion}

\end{document}

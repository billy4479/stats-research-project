\documentclass[12pt]{extarticle}

\title{Does commuting affect quality of life?}
\author{Ellero Giacomo, Porta Beatrice}
\date{a.y. 2024/2025}

\usepackage{parskip} % for no indentation
\usepackage[a4paper,margin=1.5cm]{geometry} % to adjust margins
\usepackage{bookmark} % for pdf bookmarks

\usepackage{enumitem} % for custom lists
\usepackage{float}
\usepackage{dirtytalk}
\usepackage{array}
\usepackage{makecell}

% We want numbers on subsections too
\setcounter{secnumdepth}{3}

\usepackage{amssymb}   % for varnothing and other weirds symbols
\usepackage{amsmath}   % basically everything
\usepackage{amsthm}    % for proof environment
\usepackage{mathtools} % for underbrace, arrows, and a lot of other things
\usepackage{mathrsfs}  % for "mathscr"
\usepackage{bm}        % for bold math symbols
\usepackage{physics}   % for derivatives and lots of operators
\usepackage{dsfont}    % for \mathds{1}

\numberwithin{table}{section}
\numberwithin{figure}{section}

\usepackage{cancel}    % for canceling terms
\renewcommand{\CancelColor}{\color{red}}

% Operators
\newcommand{\gt}{>}
\newcommand{\lt}{<}
\newcommand{\indep}{\perp \!\!\! \perp}

% Sets
\newcommand{\C}{\mathds{C}}
\newcommand{\R}{\mathds{R}}
\newcommand{\N}{\mathds{N}}
\newcommand{\Q}{\mathds{Q}}
\newcommand{\Z}{\mathds{Z}}

\numberwithin{equation}{section}

\begin{document}

\maketitle

\section{Introduction}

With this research we want to investigate whether there’s a difference in the academic performances
and the quality of life between students who are commuters and those who aren't.
Commuters are those people who work or study relatively far from where they live
and need to travel back and forth from their workplace/university every day.

The motive behind our research sparked because both the researchers are commuters,
alongside the majority of their friends, while in their class at Bocconi university most students
live very close to the university.
We wondered if living in the city where their University is located could be beneficial
to their well-being and also academic performance.

The intuition we had is that the time spent commuting could be better invested in studying
or doing other activities more beneficial to the overall well-being.
Moreover, public transports are often a source of stress, due to delays, strikes and discomfort.

However, the benefits of not having to commute are all mere speculations without any data to prove
that there’s an actual correlation.

All the data we collected and the scripts we used are available at
\url{https://github.com/billy4479/stats-research-project}.

\section{Data collection}

As we didn't find any existing dataset on this specific topic of research we decided
to collect our own data through a survey.

The survey, both in Italian and in English, was sent to many students all over Italy.
We managed to collect about $100$ answers, with about $60\%$ of replies being from commuters.

This is probably the greatest limitation of our project:
the sample is very small and not very diverse, which does not help providing accurate results.
Moreover, many of the respondents were unable to provide a GPA,
so that reduced the available data even further.

After collecting the data we passed it through the \texttt{merge\_data.py} script,
which merged the results from the Italian and English form and expressed the results in a more
\say{statistic friendly} way.

The complete data can be found at
\url{https://github.com/billy4479/stats-research-project/blob/master/data/merged.csv}.

\begin{table}[H]
	\begin{center}
		\begin{tabular}{|c|c|c|c|}
			\hline
			\textbf{n} & \textbf{Variable}                & \textbf{Type}                  & \textbf{Description}                           \\
			\hline
			\hline
			0          & \texttt{university}              & string                         & Name of the university attended.               \\
			\hline
			1          & \texttt{age}                     & integer                        & Age.                                           \\
			\hline
			2          & \texttt{is\_commuter}            & boolean                        & Considers themselves a commuter.               \\
			\hline
			3          & \texttt{commute\_time}           & integer $\in \{1, \dots, 5\}$  & Time taken to go to university.                \\
			\hline
			4          & \texttt{attendance}              & integer $\in \{1, \dots, 4\}$  & Percentage of classes attended.                \\
			\hline
			5          & \texttt{gpa}                     & real $\in [0,30]$              & GPA.                                           \\
			\hline
			6          & \texttt{did\_move}               & boolean                        & Did move to attend university.                 \\
			\hline
			7          & \texttt{means\_of\_transport}    & list of strings                & List of means of transport used.               \\
			\hline
			8          & \texttt{cor\_commute\_study}     & integer $\in \{1, \dots, 10\}$ & \makecell{Expects that commuting influences    \\negatively studies.}              \\
			\hline
			9          & \texttt{no\_study\_time}         & integer $\in \{1, \dots, 10\}$ & Doesn't have enough time to study.             \\
			\hline
			10         & \texttt{higher\_gpa\_if\_closer} & integer $\in \{1, \dots, 10\}$ & \makecell{Thinks they would have a better GPA  \\ if they lived closer to university.} \\
			\hline
			11         & \texttt{no\_hobbies}             & integer $\in \{1, \dots, 10\}$ & \makecell{Doesn't have enough time to spend on \\ hobbies and sports.} \\
			\hline
			12         & \texttt{stress}                  & integer $\in \{1, \dots, 10\}$ & Going to university is stressful.              \\
			\hline
			13         & \texttt{no\_sleep}               & integer $\in \{1, \dots, 10\}$ & Does not get enough sleep.                     \\
			\hline
			14         & \texttt{no\_friends}             & integer $\in \{1, \dots, 10\}$ & Not enough time to see their friends.          \\
			\hline
			15         & \texttt{no\_family}              & integer $\in \{1, \dots, 10\}$ & \makecell{Not enough time to spend             \\ with their family.}    \\
			\hline
			16         & \texttt{loneliness}              & integer $\in \{1, \dots, 10\}$ & Feels lonely.                                  \\
			\hline
		\end{tabular}
		\caption{
			Data collected through the survey.
			Integer values have been normalized in order to fint in the interval $[0,1] \in \R$.
			The data with indices $9$ through $16$ inclusive is what we called Quality of Life Indicators (QLI).
		}
	\end{center}
\end{table}

\section{Tests}

\section{Regressions}

Note that, as shown in the previous section, the GPA for commuters violates the normality assumption.
This is probably due to the low number of samples, since all the GPAs combined pass the Shapito-Wilk test.

Moreover, performing the same test on all the Quality of Life Indicators (QLI for short)
gives that they are all normally distributed.

\subsection{Linear regression on GPA}

First we tried to do a linear regression to try to estimate the GPA based on the predictors we have available.
We used as predictors
\texttt{commute\_time},
\texttt{is\_commuter},
\texttt{did\_move},
\texttt{no\_study\_time},
\texttt{higher\_gpa\_if\_closer},
\texttt{no\_hobbies},
\texttt{stress},
\texttt{no\_sleep},
\texttt{no\_family},
\texttt{no\_friends},
\texttt{loneliness}, and
\texttt{attendance}.

However, the F-test showed that none of these predictors are relevant at $\alpha_0 = 0.05$.

TODO: possiamo provare tipo step up/step down/AIC/BIC?

\subsection{Linear regression on each QLI}

Next we tried to predict each Quality of Life Indicator.
We used as predictors
\texttt{commute\_time},
\texttt{is\_commuter}, and the following variables derived from \texttt{means\_of\_transport}:
\texttt{did\_move},
\texttt{use\_foot},
\texttt{use\_bike},
\texttt{use\_bus},
\texttt{use\_metro},
\texttt{use\_tram},
\texttt{use\_train},
\texttt{use\_car}, and
\texttt{n\_means\_used}.

TODO: finire qui

\section{Conclusion}

\end{document}

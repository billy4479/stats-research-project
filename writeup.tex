\documentclass[11pt]{extarticle}

\title{\vspace{-1.5cm}Does commuting affect quality of life?\vspace{-0.5cm}}
\author{Ellero Giacomo, Porta Beatrice}
\date{}

\usepackage{parskip} % for no indentation
\usepackage[a4paper,margin=1.5cm]{geometry} % to adjust margins
\usepackage{bookmark} % for pdf bookmarks

\usepackage{enumitem} % for custom lists
\usepackage{float}
\usepackage{dirtytalk}
\usepackage{array}
\usepackage{makecell}
\usepackage{subcaption}

% We want numbers on subsections too
\setcounter{secnumdepth}{3}

\usepackage{amssymb}   % for varnothing and other weirds symbols
\usepackage{amsmath}   % basically everything
\usepackage{amsthm}    % for proof environment
\usepackage{mathtools} % for underbrace, arrows, and a lot of other things
\usepackage{mathrsfs}  % for "mathscr"
\usepackage{bm}        % for bold math symbols
\usepackage{physics}   % for derivatives and lots of operators
\usepackage{dsfont}    % for \mathds{1}

\usepackage{xcolor}
\usepackage{cleveref}
\usepackage{hyperref}
\hypersetup{
    colorlinks=true,
    linkcolor=teal,
    urlcolor=blue,
}
\urlstyle{same}

\numberwithin{table}{section}
\numberwithin{figure}{section}

\usepackage{cancel}    % for canceling terms
\renewcommand{\CancelColor}{\color{red}}

% Operators
\newcommand{\gt}{>}
\newcommand{\lt}{<}
\newcommand{\indep}{\perp \!\!\! \perp}

% Sets
\newcommand{\C}{\mathds{C}}
\newcommand{\R}{\mathds{R}}
\newcommand{\N}{\mathds{N}}
\newcommand{\Q}{\mathds{Q}}
\newcommand{\Z}{\mathds{Z}}

\numberwithin{equation}{section}

\begin{document}

\maketitle
\vspace{-1cm}
%\tableofcontents

\section{Introduction}

With this research we want to investigate whether there’s a difference in the academic performances
and the quality of life between students who are commuters and those who aren't.
Commuters are those people who work or study relatively far from where they live
and need to travel back and forth from their workplace/university every day.

The motive behind our research sparked because both the researchers are commuters,
alongside the majority of their friends, while in their class at Bocconi university most students
live very close to the university.
We wondered if living in the city where their University is located could be beneficial
to their well-being and also academic performance.

The intuition we had is that the time spent commuting could be better invested in studying
or doing other activities more beneficial to the overall well-being.
Moreover, public transports are often a source of stress, due to delays, strikes and discomfort.
However, the benefits of not having to commute are all mere speculations without any data to prove
that there’s an actual correlation.

All the data we collected along with the scripts we used and the source code for this document
are available at
\href{https://github.com/billy4479/stats-research-project}{the following GitHub repository}.

\section{Data collection and dataset}

As we didn't find any existing dataset on this specific topic of research we decided
to collect our own data through a survey.
The survey, both in Italian and in English, was sent to many students all over Italy.
We managed to collect about $100$ answers, with about $60\%$ of replies being from commuters.

After collecting the data we passed it through the \texttt{merge\_data.py} script,
which merged the results from the Italian and English form and expressed the results in a more
\say{statistic friendly} way.
The complete data can be found
\href{https://github.com/billy4479/stats-research-project/blob/master/data/merged.csv}{here}.
The final dataframe was organized accordingly to \cref{table:data}.

\begin{table}[!ht]
	\centering
	\begin{tabular}{|c|c|c|c|}
		\hline
		\textbf{n} & \textbf{Variable}                & \textbf{Type}              & \textbf{Description}                                \\
		\hline
		1          & \texttt{university}              & string                     & Name of the university attended.                    \\
		2          & \texttt{age}                     & int                        & Age.                                                \\
		3          & \texttt{is\_commuter}            & boolean                    & Considers themselves a commuter.                    \\
		4          & \texttt{commute\_time}           & int $\in \{1, \dots, 5\}$  & Time taken to go to university.                     \\
		5          & \texttt{attendance}              & int $\in \{1, \dots, 4\}$  & Percentage of classes attended.                     \\
		6          & \texttt{gpa}                     & real $\in [0,30]$          & GPA.                                                \\
		7          & \texttt{did\_move}               & boolean                    & Did move to attend university.                      \\
		8          & \texttt{means\_of\_transport}    & list of strings            & List of means of transport used.                    \\
		9          & \texttt{cor\_commute\_study}     & int $\in \{1, \dots, 10\}$ & Believes commuting negatively affects studies.      \\
		10         & \texttt{no\_study\_time}         & int $\in \{1, \dots, 10\}$ & Doesn't have enough time to study.                  \\
		11         & \texttt{higher\_gpa\_if\_closer} & int $\in \{1, \dots, 10\}$ & Thinks GPA would be higher if closer to university. \\
		12         & \texttt{no\_hobbies}             & int $\in \{1, \dots, 10\}$ & No time to spend on hobbies and sports.             \\
		13         & \texttt{stress}                  & int $\in \{1, \dots, 10\}$ & Going to university is stressful.                   \\
		14         & \texttt{no\_sleep}               & int $\in \{1, \dots, 10\}$ & Does not get enough sleep.                          \\
		15         & \texttt{no\_friends}             & int $\in \{1, \dots, 10\}$ & Not enough time to see their friends.               \\
		16         & \texttt{no\_family}              & int $\in \{1, \dots, 10\}$ & Not enough time to spend with their family.         \\
		17         & \texttt{loneliness}              & int $\in \{1, \dots, 10\}$ & Feels lonely.                                       \\
		\hline
	\end{tabular}
	\caption{
		Data collected through the survey.
		Integer values have been normalized in order to fint in the interval $[0,1] \in \R$.
		The data with indices $9$ through $16$ inclusive is what we called Quality of Life Indicators (QLI).
	}
	\label{table:data}
\end{table}

\section{Tests}
\label{sec:tests}

In order to test these hypotheses we first performed the Shapiro test to investigate the normality
of the distributions of the respective GPAs of the two samples, commuters and non-commuters.
We found that commuters' GPA did not follow a normal distribution, while non-commuters' GPA
had a normal distribution.

Each paragraph is titled after the alternate hypotheses we are trying to confirm.

\paragraph{Commuting is correlated to a lower GPA}
Since we saw that our sample is not normal,
when testing our first hypothesis we used the nonparametric Mann-Whitney U Test.
We found a $p$-value of $0.0152$, which indicates that we can reject the null hypothesis because
there is statistical evidence to support the claim that commuters have a lower GPA than non-commuters.

\paragraph{More time spent commuting is correlated to a lower GPA}
For the second hypothesis, we performed an ANOVA test
which tests for differences in means across multiple groups.
We considered both the commuters and the non-commuters sample.
The result was a $p$-value of $0.0507$, which indicates that we fail to reject the null hypothesis,
but also that, since the p-value is very close to the threshold, there could be a marginal effect.
If a stricter significance level (like $0.01$) was used,
the result would be considered non-significant.

\paragraph{Commuting is correlated to a lower quality of life}
For our third hypothesis, we first computed the total mean of the QLIs of commuters and
then investigated the normality of this mean with the Shapiro tests.
We found that the mean follows a normal distribution and therefore we proceeded with a t-test.
The result was a $p$-value of $0.8765$, so we fail to reject the null hypothesis.
This means that there is no significant evidence to suggest that the mean of commuters' QLIs
is less than that of non-commuters.
We also found that the two means of the two groups are very close to each other
($0.5171$ vs $0.4692$), which further suggests no meaningful difference.

\paragraph{Commuting using public transport is correlated to a lower quality of life}
For our fourth hypothesis, sadly we couldn’t conclude anything.
We first separated the dataset in two groups based on the means of transport,
namely private or public ones, but when performing the Shapiro test to investigate the normality
of the distribution of these two groups we encountered an error:
the sample for the private group was too small to check normality.
Therefore, we stopped and the question remains open.

\begin{figure}[!ht]
	\centering
	\subcaptionbox{GPA vs commute time.}
	{\includegraphics[width=0.3\textwidth]{./routput/graphs/boxplot_gpa_vs_commute_time.pdf}}
	\subcaptionbox{GPA vs is commuter.}
	{\includegraphics[width=0.3\textwidth]{./routput/graphs/boxplot_gpa_vs_is_commuter.pdf}}
	\subcaptionbox{Mean of QLIs vs is commuter.}
	{\includegraphics[width=0.3\textwidth]{./routput/graphs/boxplot_mean_qli_vs_is_commuter.pdf}}
	\caption{Some boxplots of our data.}
\end{figure}

\section{Regressions}

\subsection{Assumptions}

Linear regressions make some assumptions about the data which have to be checked before
running the model.

\paragraph{Normality}
First we have to verify that the independent variable follows a normal distribution.
We performed Shapiro-Wilk test on the GPA and all the QLIs which told us
with very high confidence that our data is not normally distributed.
We also made some QQ-plots to confirm graphically the results of the test.

\begin{figure}[!ht]
	\centering
	\subcaptionbox{QQ-plot for the \texttt{gpa}}
	{\includegraphics[width=0.3\textwidth]{./routput/gpa_regression/qqplot_gpa.pdf}}
	\hspace{1cm}
	\subcaptionbox{QQ-plot for \texttt{stress}}
	{\includegraphics[width=0.3\textwidth]{./routput/qli_regression/qqplot_stress.pdf}}
	\caption{QQ-plot of \texttt{gpa} and \texttt{stress}
		(taken as an example, other QLIs show a similar behavior).
	}
	\label{fig:qq-gpa-stress}
\end{figure}

As can be seen from the plots in \cref{fig:qq-gpa-stress},
our data is neither very normal nor very continuous:
it is clearly possible to see that there were $10$ options available in the QLI survey
and that most people wrote an integer GPA.

This is probably due to the very small sample size
and the samples are not \textit{too} far away from the confidence region of the QQ-plot.
However this is definitely something to keep in mind when
evaluating the effectiveness of our regressions.

\paragraph{Correlation}
The next step we took was to verify that our predictors
were not too correlated with each other as this can negatively impact the performance
of the regression.

\begin{figure}[!ht]
	\centering
	\includegraphics[width=0.5\textwidth]{./routput/correlation/correlation.pdf}
	\caption{
		Correlation matrix for all the predictors used in regressions.
		Empty cells represent correlation not significant at level $0.05$.
	}
	\label{fig:corr}
\end{figure}

As it is possible to see in \cref{fig:corr}, there are some predictors which
are very correlated with each other.
From this matrix we can extract the following information:
\begin{itemize}
	\item \texttt{commute\_time}, \texttt{is\_commuter} and \texttt{did\_move}
	      are all very correlated;
	\item \texttt{is\_commuter} is positively correlated to the use of public transport
	      and in particular to the use of trains, while negatively to going by foot;
	\item The use of the metro is positively correlated to
	      the number of means used and the use of public transport at all.
\end{itemize}

From these results we decided to exclude \texttt{is\_commuter} and \texttt{did\_move}
from the regressions as they are too correlated to \texttt{commute\_time},
which also carries more information than the other two.

\paragraph{Residuals}
To complete the checks on the assumptions of the linear regressions
we also checked that the residuals are normally distributed with constant variance.

Both these assumptions where verified for all regressions,
the first one by using Shapiro-Wilk tests and QQ-plots, the second one using scatter plots.
%(see \cref{fig:residuals}).

%\begin{figure}[!ht]
%	\centering
%	\subcaptionbox{QQ-plot for the residuals in the GPA regression.}
%	{\includegraphics[width=0.3\textwidth]{./routput/gpa_regression/qqplot_residuals.pdf}}
%	\hspace{1cm}
%	\subcaptionbox{Scatter plot for residuals in \texttt{no\_sleep} regression.}
%	{\includegraphics[width=0.3\textwidth]{./routput/qli_regression/scatter_residuals_no_sleep.pdf}}
%	\caption{A QQ-plot and a scatter plot of the residuals in two different regressions.
%		These graphs were taken as examples:
%		in other regressions we obtained the same results.}
%	\label{fig:residuals}
%\end{figure}

\subsection{Linear regression on GPA}

First we tried to do a linear regression to try to estimate the GPA
based on the predictors we have available.

We started by using almost all the predictors we had available,
from there we started removing those predictors that were correlated with each other,
according to \cref{fig:corr}, and to reduce the possibility of overfitting
we also performed a stepwise regression to eliminate predictors in excess.

We ran the stepwise regression both forwards and backwards,
using the AICc penalty criterion, which is a modified version of AIC with an added penalty term
for the size of the dataset.
This was particularly useful since small datasets are more prone to overfitting.
\begin{equation}
	{\operatorfont AICc} =
	\underbrace{2k + 2 \ln (\hat L)}_{\operatorfont AIC}
	+ \underbrace{\frac {2k^{2}+2k}{n-k-1}}_{\text{extra penalty term}}
\end{equation}

We algorithm settled at a model containing the following predictors:
\texttt{commute\_time},
\texttt{higher\_gpa\_if\_closer},
\texttt{no\_study\_time},
\texttt{no\_hobbies},
\texttt{stress},
\texttt{no\_sleep},
\texttt{no\_family},
\texttt{no\_friends},
\texttt{loneliness}, and
\texttt{attendance}.

Running the linear regression with these predictors gave as result that almost none of them
are significantly different from $0$, with the exception of
\texttt{commute\_time} and \texttt{loneliness} which both had a $p$-value of $0.07$.
The model had a very low adjusted $R^2$ around $0.15$.

However, running an analysis of variance on the model showed that both
\texttt{commute\_time} and \texttt{no\_study\_time} are significant
at explaining the variance of the model, with both $p$-values $<0.05$.

\subsection{Linear regression on each QLI}

Next we tried to predict each Quality of Life Indicator.
We used the same stepwise algorithm on each indicator we wanted to predict,
which gave us a moderately size model for the linear regression.
The full model we started from used the following predictors:
\texttt{commute\_time},
\texttt{is\_commuter}, and the following variables derived from \texttt{means\_of\_transport}:
\texttt{did\_move},
\texttt{use\_foot},
\texttt{use\_bike},
\texttt{use\_bus},
\texttt{use\_metro},
\texttt{use\_tram},
\texttt{use\_train},
\texttt{use\_car}, and
\texttt{n\_means\_used}.

Here again all regressions have a very low adjusted $R^2$ value, hovering around $0.1$ for all QLIs.
We will report just the most notable findings between all these regressions.

\paragraph{Stress}
This is the regression that gave best results:
we got that
\texttt{commute\_time},
\texttt{use\_public\_transport},
\texttt{n\_means\_used}
are all significant at level $0.05$ with positive coefficients,
while \texttt{use\_car} is also significant but with negative coefficient.

\paragraph{Higher GPA if closer}
The linear regression told us that none of our predictors had coefficients
significantly different from $0$, however, running ANOVA gave that with
very high confidence ($p$-value $= 0.0009$) that \texttt{commute\_time}
causes variance in the result, which sadly was not encapsulated in our linear model.

\paragraph{No sleep}
In this regression we found that the intercept is very significant,
hence lack of sleep is a common problem among all students.
Moreover, \texttt{use\_train} was also significant with $p$-value of $0.00018$
and positive coefficient, which makes sense as trains usually have a
lower frequency compared to other means of transport and it is often required
to wake up very early in order to catch the right one.

\paragraph{Other QLIs}
In other regressions we got very bad $R^2$ values and none of our predictors
were found significant.
In some cases ANOVA revealed that some predictors such as \texttt{commute\_time}
are the cause for variance in the result, but the linear regressions were not able
to model it.

\section{Conclusions, limitations and further questions}

\paragraph{Conclusions}
Our tests showed that there is a correlation between GPA and commuting,
in particular commuting results in a lower GPA.
However, this was very hard to model into a linear regression, giving models with very low
predictive power.
For QLIs tests showed no significant difference between commuters and non commuters.
This is also reflected in the linear regressions: they all have incredibly low predictive power
and almost none of our predictors were significant.

We can justify the first result by thinking that since commuters spend a larger time on
means on transport, they have to subtract this time from their \say{study time},
since studying while traveling is often not possible due to a noisy environment,
discomfort and even if possible would probably not result in a very high level quality of study
(but this should be further researched).
Moreover, after arriving home from a long travel many students feel too tired in order to focus
and proficiently study at home.
This could lead to a more in depth study on the effects of commuting,
namely on how traveling impacts concentration and energy drainage.

For the second result, this could be due to the fact that while commuters may feel stressed
because of the uncomfortable means of transport, strikes and unforeseen problems they may face,
non-commuters often had to relocate to the city in which they study,
which means leaving behind their family and friends.
This means they often feel lonely and nostalgic, but also that they need to invest time
to take care of their house and other business which may result in high stress levels.

\paragraph{Limitations}
We faced, in our study, many limitations, the first of all being that we dealt with
a very limited dataset.
This resulted in our data not being enough to conclude some tests and
probably also worsened the results of our regressions.

Moreover, our dataset was also disproportionate: over half of the participants
of our survey were commuters and our sample was mostly from students living near Milan,
which is also quite restrictive.
It would be ideal to replicate our study on a larger scale which involves an equal part
of commuters and non-commuters from a larger region.

Moreover, our survey also had very limited resolution on the \texttt{commute\_time} variable:
in the survey people had just $4$ options to select from; looking back at it, it would have been
much better to ask for the commute time in minutes as this was the main predictor for our study.

\paragraph{Further questions}
There are more questions that could be answered: one could be how GPA impacts a student’s life.
For example, a higher GPA could either result in a happy student or a very stressed student,
who had to sacrifice their private life in order to achieve the academic result.
Another question could be the reverse, namely if having a stressful life result in a lower GPA.

We didn’t investigate these questions because out of the scope of our research,
but we believe that they could lead to interesting result and spark a very important conversation
on student’s mental health and how school affects it.

\end{document}
